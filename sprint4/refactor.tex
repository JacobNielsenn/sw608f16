\section{Code refactoring}
In this section we describe the changes made to the code in the fourth sprint.

\subsection*{Hashing of MAC-Address}
Due to a request from the heatmap group, to be able to track people by some unique value we have elected to hash the MAC-Address of all people not on the network rather than relapsing it with "Obfuscate". In \cref{sec:secure} we explored the possibility of doing so, by using Javas hashCode(), which is implemented in the String library. However hashCode() returns only an integer which gives a higher risk of having overlap when hashing. Instead we chose to use the guava-19 library which returns a much longer value as a string, by doing so it allows for hashed MAC-Address to contain letters making it more difficult to brute-force. The salt applied is a random string for which a new is generated every twelve hour. 

The applied function can be seen in \cref{lst:ourhash} where in line 1 the MAC-Address is concatenated with the random generated salt-value. In line two the new string is being hashed using guava-19's sha256 at last in line 3 the MAC-Address is set to the hashed value. The hashed value is a 64 signs long containing both letters and numbers.

\begin{lstlisting}[caption={Hashing a MAC-Address},label={lst:ourhash},language=inc_Java]
oldMacAddress = oldMacAddress + salt;
String hash = Hashing.sha256().hashString(oldMacAddress, 
	StandardCharsets.UTF_8).toString();
item.setMacAddress(hash);
\end{lstlisting}

\subsection*{Adding servers to client}

\subsection*{Accessors}