\section{Loaded Acceleration Test} %loaded accel test
This test will use the same test program to test how the motors reacts to the extra workload associated with the rig. The motor will be connected to the rig. This is done in order to compare the results with the previous acceleration test and to get an understanding of its perform in practice. the hypothesis is based on the fact that the increased workload makes the motors pull more weight and will therefore need more time to reach its maximum speed, but the workload will also make the motor vibrate less and should therefore lessen the spread of end results.

\subsection{Hypothesis} The motor will use longer time and turn further before it reaches its maximum speed and the spread in end results will however be smaller, compared to the tests performed in free space.

\subsection{Test Procedure}
In order to test the loaded acceleration of a motor, the same specialised test program, as used in the acceleration test, has been constructed to record the data and to uniform the test procedure. The program is written to execute the test 25 times, and therefore the test only has to be executed 4 times, to conduct the test 100 times, as specified in the test structure. To execute the test, the following list must be followed:

\begin{itemize}
	\item Upload test software to \gls{nxt}
	\item Attach the test motor to the rig
  	\item Attach the \gls{nxt} motor to be tested, to the \gls{nxt} brick
	\item Execute test software Y times
	\item Record test results
\end{itemize}

\subsection{Data Analysis}
Based on the test data collected during the loaded acceleration test, it is possible to draw certain conclusions. The data is presented in \cref{fig:td_l_accel} presents \emph{motor 2} and \emph{motor 4} as being the fastest. The motors with the lowest standard deviation and is \emph{motor 1} and \emph{motor 2}.

Since there is a conflict between choosing between \emph{motor 2} and \emph{motor 4} the conclusion from \cref{sec:accel_test} have been revisited. It concluded that \emph{motor 1} and \emph{motor 4} was the best choice, and since both of these motors is candidates to be chosen in this test as well, it is concluded that thy are the most suitable.

\subsection{Test Conclusion}
Based on the data analysis of the data collected during the loaded acceleration test and the conclusion from the unloaded acceleration test, it can be concluded that \emph{motor 1} and \emph{motor 4} are most suitable for usage in the system.



\section{Acceleration Test} \label{sec:accel_test} %Accel test
This test is meant to determine the acceleration of the motors going from 0 to maximum angular Velocity. The motors are tested individually, to ensure more valid results, where the execution of one motor will not affect the result of the other motors.

\subsection{Hypothesis} If one motor is instructed to reach maximum speed, the time for this action, will be the same across multiple tests on same motor.

\subsection{Test procedure}
In order to test the acceleration of a motor, a specialised test program has been constructed to record the data and to uniform the test procedure. The program is written to execute the test 25 times, and therefore the test only has to be executed 4 times, to conduct the test 100 times, as specified in the test structure. To execute the test, the following list must be followed:

\begin{itemize}
	\item Upload test software to \gls{nxt}
	\item Attach the \gls{nxt} motor to be tested, to the \gls{nxt} brick
	\item Execute test software Y times
	\item Record test results
\end{itemize}

\subsection{Data Analysis}
The tests have other peeks which is immensely higher or lower than the average, an example of this can be seen in acceleration test 4 where the average is $0.66 degrees/millisecond$ but $19\%$ of the results creates a peek between the 103-105 degree mark which has an average of $0.58 degrees/millisecond$. This result is considered to be affected by the uncertainties in the test set-up mentioned in \cref{sec:tech_uncertain}. Another reason for the test result to have these immensely different peeks that is not covered by the uncertainties, is how the test program functions. The test program asks the motor every 20'th millisecond if the degrees it have turned is equal or lower to how many it turned 40 milliseconds ago, which will indicate if it is still accelerating or if it have reached its maximum speed. The motor counts the degrees it have turned in rational numbers which means that the comparisons made between the degrees turned will only account for whole numbers. This can make the motor continue to rotate the rotor for an additional 20 milliseconds even after it had actually peeked at its maximum speed. This can happen for the reason that it could have turned 20.9 degrees in one read and 21.0 in the next read which would indicate an acceleration even though it might just be variations in its speed. This can be seen in the test results as many of the greater peeks have minor peeks 20-40 milliseconds afterwards. The phenomenon can be seen in the first acceleration test where the 280 milliseconds mark had 37 end results and 20 milliseconds afterwards at the 300 mark another 23 end result created a new peek. The average degrees/millisecond is close to each other for those two peeks at 0.71 and 0.72 which indicates that the average speed were very close or maybe even equal.

The phenomenon will likewise happen in the other direction if what is an slow acceleration is seen as the motor now rotating at constant speed e.g. it goes from 19.0 to 19.9 degrees from one read to the next. Consequential this could mean that the test stops earlier than it should because it believes that the motor have stopped accelerating before it actually did.

The standard derivation is greater than what is seen in the brake tests where all of the derivations is below 1.0. The standard derivation is greatly influenced by the end result which lies far below or above the average of a test. This can be seen in \cref{fig:td_m3_accel} where one result is in the $+700$ millisecond range and in \cref{fig:td_m1_accel} where one result is before the 100 degrees mark. These are probably due to mistakes when deciding if one motor have stopped accelerating or not, as described earlier in this section, as it seems unlikely that the motor reaches its top speed before $1/5$ of a second has passed when that result is compared to the peeks in the data.

These mistakes explains why the results differs as much as they do from the hypothesis which predicted that the time it would take to reach maximum speed would be the same across tests on the same motor.

Based on the test data collected during the acceleration test, it is possible to draw certain conclusions. The data is presented in \cref{fig:td_accel} presents \emph{motor 1} and \emph{motor 4} as being the fastest and possesses the lowest standard deviation and by such most suitable for usage in the system.

\subsection{Test Conclusion}
Based on the data analysis of the data collected during the acceleration test it can be concluded that \emph{motor 1} and \emph{motor 4} are most suitable for usage in the system.

