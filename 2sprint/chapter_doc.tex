\chapter{Sprint 2} \label{cha:sprint2}
In this chapter we describe the work to be done in the second sprint. 

Since we have chosen to utilise MSE, which is available for us through AAU, the main goal for this sprint is to create a client responsible for fetching data from MSE and sending it to the aSTEP database. We will also focus on testing and thoroughly documenting existing and new code. Furthermore, we will look into the possibilities of finding geographical locations rather than relative positions within a building. Finally, we plan on describing our collaborations with other groups working on the aSTEP multi-project, as can be seen in \cref{sec:collaboration}, \nameref{sec:collaboration}.

\section{Fetching Data From Cisco} \label{sec:fetch_data}
This section aims to build a client that is able to communicate with the Cisco MSE services and the Database server. As mentioned previously, it is expected to communicate with several Cisco MSE services at a time. The client has to be built with the mindset that it will be communicating with external services that we do not have direct control of. Consequently we will have to handle external errors and unexpected behaviour without shutting the system down. 

The client has to afford the flexibility of maintaining a list of connected services during run-time, such that a new Cisco MSE service can be added and removed without requiring a restart. We call this list the $fetch list$. Furthermore, we wish to impose the requirement that a Cisco MSE service has to be online and responsive when added to the list in order to avoid faults caused by erroneous input.

In this initial design and implementation of the client we do not wish to focus on security nor efficiency as they add another layer of complexity. Hence, certain security and efficiency-related aspects have been omitted in favour of robustness and flexibility. In particular, handling sensitive data such as IP addresses and passwords is something that has to be reconsidered in the future.

\subsection*{Creating robustness}
In order to make a robust client we have to make sure that it can handle erroneous input and external errors when connecting to Cisco MSE services. This is largely handled by the $try...catch$ keywords in Java, which allow the executing program to catch exceptions during runtime and execute code according to the exception without exiting. In particular, a common exception type when working with networking is the $IOException$, which is a general term for the group of exceptions that can be encountered when performing IO-operations such as connecting to an IP address. There are several different exceptions that can be thrown when sending a request to a service, depending on the state of the service. Consequently we use $try... catch(IOException)$ when performing networking requests, as such exceptions  have to be handled in the same way no matter the subtype.

\begin{lstlisting}[caption={Failed Connection snapshot},label={lst:failed_connect},language=inc_Java, mathescape]
private void failedConnection(){
	$\vdots$
    while(!attemptSucceeded){
        timeMilis = System.currentTimeMillis();
        if(tries < 36){
            try {
                sleep(timeMilis + timeintervalSec*1000 - System.currentTimeMillis());
            } catch (InterruptedException e) {
                e.printStackTrace();
            }
        }
        $\vdots$
        else if(tries==48){
        	retry = false;
            return;
        }
        try {
            CollectAllClients(_username, _password,_ip);
        } catch (IOException e) {
            tries++;
            continue;
        }
        attemptSucceeded = true;
    }
}
\end{lstlisting}

When attempting to connect to a service and an $IOException$ is thrown, the program changes state and the method in \cref{lst:failed_connect} is executed. An example of the exception catch can be seen on line 19. Instead of fetching data and sending it to the database at an interval, the program now simply attempts reconnecting to the erroneous service. This is done at an interval, first every 10 minutes and later every 30 minutes until 12 hours have passed. After this time period the service is disregarded, and it is removed from the fetch list.

\subsection*{Creating flexibility}
We wish to afford the possibility of adding services to the fetch list during runtime. We can implement this functionality with a RESTful service while using HTTP POST requests to communicate with this service. When receiving a POST request, the service has to verify the parameters to ensure that the data received is of the correct form. 

\begin{lstlisting}[caption={Adding a server to the fetch list},label={lst:add_server},language=inc_Java, mathescape]
server.createContext("/api/add/server", httpExchange -> {
    if(VerifyHeaders(httpExchange)){
        Headers headers = httpExchange.getRequestHeaders();
        List<String> username = headers.get("Username");
        List<String> passphrase = headers.get("Password");
        List<String> url = headers.get("url");
        if(validIP(url.get(0)) && IsServerOnline(url.get(0))){
            AddIP(url.get(0), username.get(0), passphrase.get(0));
            WriteToFile();
        }
    }
});
\end{lstlisting}
\Cref{lst:add_server} shows the essential functionality executed when a request is received on the URI $/api/add/server$. On line two we call the function VerifyHeaders, which goes through the parameters of the POST request and ensures that IP address, username and password are not empty. If the request passes this test, we then test on line 7 whether the IP address provided is valid by comparing it with a regular expression and attempting to open a connection to the service. The method $IsServerOnline()$ returns true if any response code is returned by the server. If the HTTP request passes all these tests we add the IP address to the fetch list.

In order to store program configuration, we save the IP addresses and user information in a file. This is particularly helpful for debugging purposes, as we do not need to reinsert information into the program on start up. Naturally, storing sensitive information such as passwords in a file is a bad idea. However, the client is intended to have an uptime of close to 100\%, and it is as such not a necessity to store sensitive information in secondary memory in a final release. Furthermore, as the focus shifts towards security we will have to find a way to encode and store sensitive data to ensure that intruders cannot retrieve nor use any information they may find.

The fetch list is handled by creating a fetch thread for each element. This thread is tasked with the retrieval of information, obfuscation and dispatching the information to the database. In between this sequence of operations the thread will suspend its operations by sleeping, thus making processor time available for other threads. If a connection fails repeatedly for 12 hours, the thread responsible will simply die. Using threads allow for concurrent connections with Cisco MSE services and allows the underlying operating system and compiler to optimise the program to utilize several processor cores.
\section{Software Quality Assurance}\label{sec:unit_test}
To ensure correctness of the code and supply future developers with a better understanding of our code, all code has been documented and tested where it is important to ensure correct functionality. This is seen as a necessity as there are nine other project groups depending on the code we produce.

This report seeks to explain design decisions and what the purpose of the system is as a whole, whereas the usage of individual methods is documented with unit tests, and the functionality of the individual methods is documented with comments.

\subsection{Code Documentation} \label{sec:code_documentation}
We have put in additional effort to document the code for future development. This has been done by explaining all non-trivial methods in the program. Specifically, for each method we have described the general functionality as well as the method signature. In addition, links are created between methods calling classes or other methods. 

After a presentation of Java-Doc at the super-group meeting on March 31, Java-Doc was elected the standard tool for code documentation in the astep project.
As we finish documenting segments of the code, we generate a Java-Doc, which is a PDF document listing methods and their description, and upload it to the Wiki page on GitLab for other members of the project to see. The links in the comments are transferred to the Java-doc, enabling interactive documentation. 
Doing this is a continuous process and will be continued throughout the entire project, whenever code undergoes change as well as when additions are made.
\sfx{Make sure newest code is documented properly.}
\sfx{Write about testdriven development in discussion.}

\subsection{Code Testing}
Many of the aStep components communicate with each other by calling methods or sharing data. These dependencies introduce a requirement for correctness of code across the project. Consequently, unit tests are created to ensure functionality behaves according to the intention. The tests can be used to simulate the characteristics of the usage of the code and ensure that edge cases and unintended uses are handled properly. 

Changes in an implemented component may have unforeseen obscure consequences on other components because of indirect dependencies. When such an error occurs it is helpful to have tests that catch them before the changes are shared with everyone else in the project. This is ensured by use of a common GitLab server, such that whenever a change is pushed, the entire project is compiled and all the tests will be processed. The build will be marked as erroneous if the compilation or a test fails. This increase the robustness of the program, granted that unit test have been written for the program.

The code developed in this project will be further improved and expanded on by future students. Unit tests provide additional support for the documentation and facilitates understanding the code.
By writing tests for the essential parts of the code, future developers will have an easier time understanding and developing the project as well as maintain it. 

The tests across the project have been made using the testing library JUnit\cite{junit4}, which was decided at a super group meeting. The system has been supplied with tests after the implementation of each class. Each complex method in a class is tested using the black-box method, where actual output is compared to the expected output on a predefined input.

\begin{lstlisting}[caption={Testing ContinuesPuller},label={lst:test_continuespuller},language=inc_Java, mathescape]
@Test
public void testContinuesPuller1() throws Exception {
    CiscoPuller testPuller = new CiscoPuller(localIP, "test", "works");
    assertTrue("Testing 'ContinuesPuller' with a positive integer",
            testPuller.ContinuesPuller(localIP, 2));
    assertFalse("Testing 'ContinuesPuller' with a negative integer",
            testPuller.ContinuesPuller(localIP, -1));
    assertFalse("Testing 'ContinuesPuller' with 0 value integer",
            testPuller.ContinuesPuller(localIP, 0));
}
\end{lstlisting}
\Cref{lst:test_continuespuller} shows how unit testing is performed on the method $ContinuesPuller()$. The method is tested on the second parameter, which dictates how many requests are sent to a server. First, a service is created to simulate an external server, as seen on line 3. After this initialisation, the method is calling with parameters indicating two requests, minus one request and zero requests. The former is expected to be true, whereas the latter two are expected to return false, and the test passes if this is the case. 
\section{Geographical Locations} \label{sec:geo_coordinates}
There are three formats of geographical coordinates; degrees minutes seconds, degrees decimal minutes and decimal degrees\cite{geo_types}. These are represented at such:
\begin{itemize}
	\item Degrees minutes seconds(DMS): 10\degree 12' 30''
	\item Degrees decimal minutes(DDM): 10\degree 12.5'
	\item Decimal degrees(DD): 10.2083\degree
\end{itemize}

The listing below shows the conversion between the three different types of geographical coordinates. The conversions between DMS and DD comes from\cite{geo_converter} and the remaining are derived based on these formulas. All the calculations are based on equivalent values and the results of each conversion should therefore be the input of another type.


\begin{itemize}
	\item \textbf{Degrees minutes seconds on input 10\degree 12' 30''}
	\begin{itemize}
		\item To degrees decimal minutes ($DDM$)
		\begin{itemize}
			\setlength\itemsep{0.00005em}
			\item $ d = 10\degree $
			\item $ m = 12' + (30/60)'$
			\item $DDM = d \cdot m = 10\degree 12.5' $
		\end{itemize}
		\item To decimal degrees ($DD$)
		\begin{itemize}
			\setlength\itemsep{0.00005em}
			\item $ d = 10\degree $
			\item $ decM = \frac{12}{60}\degree + \frac{30}{3600}\degree = 0.2083\degree$
			\item $ DD = d + decM = 10.2083\degree $
		\end{itemize}
	\end{itemize}
	\item \textbf{Degrees decimal minutes on input 10\degree 12.5'}
	\begin{itemize}
		\item To decimal minutes seconds ($DMS$)
		\begin{itemize}
			\setlength\itemsep{0.00005em}
			\item $ d = 10\degree $
			\item $ m = \floor{12.5}' = 12' $
			\item $ s = (12.5 \mod 1) * 60'' = 30''$
			\item $ DMS = d \cdot m \cdot s = 10\degree 12' 30''$
		\end{itemize}
		\item To decimal degrees ($DD$)
		\begin{itemize}
			\setlength\itemsep{0.00005em}
			\item $ d = 10\degree $
			\item $ dec = \frac{12.5}{60}\degree = 0.2083\degree$
			\item $ DD = d + dec = 10.2083\degree$
		\end{itemize}
	\end{itemize}
	\item \textbf{Decimal degrees on input: dd = 10.2083\degree}
	\begin{itemize}
		\item To degrees minutes seconds ($DMS$)
		\begin{itemize}
			\setlength\itemsep{0.00005em}
			\item $ d = dd \mod 1 = 10\degree $
			\item $ md = (dd - d) * 60' $
			\item $ m = md \mod 1 = 12' $
			\item $ s = (md - m) * 60'' = 30''$
			\item $ DMS = d \cdot m \cdot s = 10\degree 12' 30''$ 
		\end{itemize}
		\item To degrees decimal minutes ($DDM$)
		\begin{itemize}
			\setlength\itemsep{0.00005em}
			\item $ d = dd \mod 1 = 10\degree $
			\item $ dm = (dd - d) * 60' = 12.5' $
			\item $ DDM = d \cdot dm = 10\degree 12.5'$
		\end{itemize}
	\end{itemize}
\end{itemize}

It has been requested that the data we supply for the applications are in geographical coordinates.
As the coordinate received from Cisco MSE are in feet relative to the fixed $(0;0)$ point, it would be possible to calculate the geographical location of a device by knowing the coordinate of $(0;0)$ and the angle of the map compared to longitude and latitude. 

However, by using Cisco it is not necessary to perform these calculations as there is an option of getting the positions in geographical coordinates. This requires at least three geographical coordinates on each map to be calculalted. By having geographical coordinates for at least three points, Cisco can calculate the position of each device based on the relative coordinate in comparison to the inserted locations\cite{geo_cisco}. 
Cisco can output geographical coordinates in either the DMS of the DD format\cite{cisco_geo_type}. It has been stated from other astep groups there is a preference for the DD format, but since future applications may want the DMS or DDM format, we will implement functionality to support the conversion. \ofx{we may have to correct this later on based on what we get from Cisco}

%http://www.cisco.com/c/en/us/td/docs/solutions/Enterprise/Borderless_Networks/Unified_Access/CMX/CMX_Prime.html#pgfId-1019702

\section{Sprint Evaluation}\label{sec:s2_eval}
To perform an evaluation of the sprint we iterate over the sprint goals and comment on whether or not they were fulfilled.

First, we wanted to fetch the data from MSE and send it to the aSTEP database. This goal has only partly been achieved because of several factors. First, it requires an agreement on the schemas of the database for the multi-project, which has not been settled yet. Secondly, we found that the intermediate server requires a few changes, which have not been possible to implement as the MSE administrator has been unavailable. However, we built a largely finished client service capable of fetching data from different sources and sending it to a database. We cover the implementation of the client in \cref{sec:fetch_data}.

We implemented a significant amount of unit tests to ensure correctness of our code. Furthermore, we thoroughly documented the code to decrease future development and maintenance costs. The unit tests are described in \cref{sec:unit_test}.

We looked into the possibilities of retrieving geographical data in \cref{sec:geo_coordinates}, however this is delayed as we are waiting for the local MSE administrator to allow for it.

Finally we have described several collaborations we participated in during the sprint. These range from server administration to various meetings with different groups. This can be found in \cref{sec:collaboration}. 

In conclusion, the goals for the second sprint have been partially fulfilled and further work will therefore be needed, we will aim to finish these shortcomings in the next sprint.