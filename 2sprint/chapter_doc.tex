\chapter{Sprint 2} \label{cha:sprint2}
In this chapter we describe the work done in the second sprint. 

Since we have chosen to utilize Cisco MSE, which is available to us through Aalborg University, the main goal for this sprint is to create a client responsible for fetching data from Cisco MSE to the aStep database, as it can be seen in \cref{fig:cisco_systems}. We will also focus on testing and thoroughly documenting existing and new code. Furthermore, we will look into the possibilities of finding geographical locations rather than relative positions within a building. Finally, we plan on describing our collaborations with other groups working on the aStep project, as can be seen in \cref{sec:collaboration}, \nameref{sec:collaboration}.

\section{Geographical Locations}
It have been requested that the data we supply for the apps are in geographical coordinates, there have therefore been looked into possibilities of doing so.
As the coordinate given from Cisco are in feet away from the fixed ${0;0}$ point, it would be possible to calculate the geographical location of a device by knowing the coordinate of ${0;0}$. However by using Cisco it is not necessary to do so since there in Cisco is a option of getting the positions in geographical coordinates. This requires at least three and at most 20 coordinates on each map to known. By knowing at least three points Cisco can calculate the position of each device based on the relative coordinate in comparison to the known locations\cite{geo_cisco}.


%http://www.cisco.com/c/en/us/td/docs/solutions/Enterprise/Borderless_Networks/Unified_Access/CMX/CMX_Prime.html#pgfId-1019702
\section{Unit Testing}
To supply the next year student, that will continue this project, with a better understating of our code and what the essence of it is, all methods have been tested and documented. 
The test are also necessary since there is 9 other groups that is depending our code to be working.

\subsection{Testing Purpose}
%Next year student
The code that have been developed in this project will be further improved and expanded by the next year students in order to better understand and change the code as future requirements arises for the indoor service. 
By writing tests to all of the code, give these developers a better start when they get their hands on the project. 
This report will help them understand why some design decisions have been made and what the purpose of the system is as a whole, but what the individual methods are supposed to do will not be covered in its entirety. 
By testing the methods with tests that checks if an given input returns an expected result, will help the developers to get a clearer understanding of what the methods are supposed to do. 
This understanding are further enhanced by the documentation that also describes the methods as explained in Section \ref{sec:code_documentation} \nameref{sec:code_documentation}. 

%Debugging intern and extern
This project involves the collaboration between 10 groups to develop the aStep platform. 
Many of the aStep components communicates with each other by calling methods or sharing data.
Tests are therefore beneficial as they can be used in debugging situations when big changes are made to the aStep system. 
Sometimes a change in one component will break other components in sometimes unforeseen ways such as indirect dependencies.
When such an error occurs it is helpful to have tests that catch them before the changes are shared with everyone else in the project. This is ensured by our gitlab server, were each time a commit is made the gitlab compiles the entire project and runs all the test, if any of the fail the push will not go through.

%Correctness
The tests will also be used to ensure the system provides correct functionality in the form of what they output.


%Intern understanding group and project


\subsection{Testing Method}
Junit and Java
After code
Gitlab docker and images


\section{Fetching Data From Cisco} \label{sec:fetch_data}
This section aims to build a client that is capable of communicating with several MSE services and the Database server. As mentioned previously in \Cref{subsec:system_hierarchy}, it is expected to communicate with several Cisco MSE services at a time. The client has to be built with the mindset that it will be communicating with external services that we do not have direct control of. Consequently we will have to handle external errors and unexpected behaviour without shutting the system down. 

The client has to afford the flexibility of maintaining a list of connected services during runtime, such that a new Cisco MSE service can be added and removed without requiring a restart. We call this list the \emph{fetch list}. Furthermore, we wish to impose the requirement that an MSE service has to be online and responsive when added to the list in order to avoid faults caused by erroneous input.

In this initial design and implementation of the client we do not wish to focus on security nor efficiency as they add another layer of complexity. Hence, certain security and efficiency-related aspects have been omitted in favour of robustness and flexibility. In particular, handling sensitive data such as IP addresses and passwords is something that has to be reconsidered in the future. The dimension of robustness is a measurement of how resilient a program is to erroneous input whereas the aspect of flexibility translates to the degree a program can adapt to changes. For this specific program, flexibility manifests itself in being able to handle an external service's sporadic behaviour during runtime and being capable of handling additional external MSE services during runtime.

\subsection*{Creating robustness}
In order to make a robust client we have to make sure that it can handle erroneous input and external errors when connecting to MSE services. This is largely handled by the $try...catch$ keywords in Java, which allow the executing program to catch exceptions during runtime and execute code according to the exception without exiting. In particular, a common exception type when working with networking is the $IOException$, which is a general term for the group of exceptions that can be encountered when performing IO-operations such as connecting to an IP address. There are several different exceptions that can be thrown when sending a request to an MSE service, depending on the state of the service. Consequently we use $try... catch(IOException)$ when performing networking requests, as resulting exceptions have to be handled in the same way no matter the subtype.

\begin{lstlisting}[caption={Failed Connection snapshot},label={lst:failed_connect},language=inc_Java, mathescape]
private void failedConnection(){
	$\vdots$
    while(!attemptSucceeded){
        timeMilis = System.currentTimeMillis();
        if(tries < 36){
            try {
                sleep(timeMilis + timeintervalSec*1000 - System.currentTimeMillis());
            } catch (InterruptedException e) {
                e.printStackTrace();
            }
        }
        $\vdots$
        else if(tries==48){
        	retry = false;
            return;
        }
        try {
            CollectAllClients(_username, _password,_ip);
        } catch (IOException e) {
            tries++;
            continue;
        }
        attemptSucceeded = true;
    }
}
\end{lstlisting}

When attempting to connect to a service and an $IOException$ is thrown, the program changes state and the method called failedConnection, which can be seen in \cref{lst:failed_connect}, is executed. Instead of fetching data and sending it to the database at an interval, the program now simply attempts reconnecting to the erroneous service. This is attempted on line 18, and is done at an interval, first every 10 minutes and later every 30 minutes until 12 hours have passed, as decided by sequentially incrementing the variable \emph{tries}. This variable is incremented as more reconnects are attempted. After 48 tries, equivalent to 12 hours, the service is disregarded, and it is removed from the fetch list. Lines 6 and 19 are examples of the usage of the $try...catch$ keywords.

\subsection*{Creating flexibility}
We wish to afford the possibility of adding services to the fetch list during runtime. We can implement this functionality with a RESTful service while using HTTP POST requests to communicate with this service. When receiving a POST request, the service has to verify the parameters to ensure that the data received is of the correct form. 

\begin{lstlisting}[caption={Adding a server to the fetch list},label={lst:add_server},language=inc_Java, mathescape]
server.createContext("/api/add/server", httpExchange -> {
    if(VerifyHeaders(httpExchange)){
        Headers headers = httpExchange.getRequestHeaders();
        List<String> username = headers.get("Username");
        List<String> passphrase = headers.get("Password");
        List<String> url = headers.get("url");
        if(validIP(url.get(0)) && IsServerOnline(url.get(0))){
            AddIP(url.get(0), username.get(0), passphrase.get(0));
            WriteToFile();
        }
    }
});
\end{lstlisting}
\Cref{lst:add_server} shows the essential functionality executed when a request is received on the URI $/api/add/server$. On line two we call the function VerifyHeaders, which goes through the parameters of the POST request and ensures that the new service has a nonempty IP address, username and password. If the request passes these tests, we then examine on line 7 whether the IP address provided is valid by comparing it with a regular expression and attempting to open a connection to the service. The method $IsServerOnline()$ returns true if any response code is returned by the server. If the HTTP request passes all these tests we add the IP address to the fetch list.

In order to store program configuration, we save the IP addresses and user information in a file. This is particularly helpful for debugging purposes, as we do not need to reinsert information into the program on each start up. Naturally, storing sensitive information such as passwords in a file is a bad idea. However, the client is intended to have an uptime of close to 100\%, and it is as such not a necessity to store sensitive information in secondary memory in a final release. Furthermore, as the focus shifts towards security we will have to find a way to encode and store sensitive data to ensure that intruders cannot retrieve nor use any information they may find.

The fetch list is handled by creating a fetch thread for each element. This thread is tasked with the retrieval of information, obfuscation and dispatching the information to the database. In between this sequence of operations the thread will suspend its operations by sleeping, thus making processor time available for other threads. If a connection fails repeatedly for 12 hours, the responsible thread will simply die. Using threads allow for concurrent connections with MSE services and allows the underlying operating system and compiler to optimise the program to utilize several processor cores.
\section{Data Storage} \label{sec:data_store}
Talk about what data we have decided should be send to the database.

\section{Sprint evaluation}
To perform an evaluation of the sprint we iterate over the sprint goals and comment on whether or not they were fulfilled.

First, we wanted to fetch the data from Cisco MSE and send it to the aSTEP database. This goal has only partly been achieved because of several factors. First, it requires an agreement on the uniformity of the data for the whole project, which has not been settled yet. Secondly, we found that the intermediate server requires a few changes, which have not been possible to implement as the Cisco MSE administrator has been unavailable. However, we built a largely finished client service capable of fetching data from different sources and sending it to a database. We cover the implementation of the client in \cref{sec:fetch_data}.

We implemented a significant amount of unit tests to ensure correctness of our code. Furthermore, we thoroughly documented the code to decrease future development and maintenance costs. The unit tests are described in \cref{sec:unit_test}.

We looked into the possibilities for storing geographical data, and we are curently awaiting the local Cisco administrator to allow for this.

Finally we have described several collaborations we participated in during the sprint. These range from server administration to various meetings with different groups. This can be found in \cref{sec:collaboration}. 

In conclusion, the goals for the second sprint have been partially fulfilled and further work will therefore be needed.
