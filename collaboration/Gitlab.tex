\section{GitLab}
From the aSTEP meeting that toke place 17.02.2016 it was decided that a few people across the groups should take responsibility for maintaining the gitlab server. Two members of our group volunteered for this task. The section purpose is to describe the work that our group has done related to GitLab.

We toke on the task of updating the GitLab that was running on our aSTEP virtual server, this will from now on be refereed to as aSTEP server. The magnitude of this task involves installing a virtual server on one of our groups computer to emulate the GitLab update to ensure that if it fails to update, we will not course any problems to the aSTEP server running GitLab. Later in the process we discovered some problems with this approach.

Installing the ubuntu-server version 13.04 went well, creating the backup image of the gitlab server was a little more difficult, but mainly because of our inexperience with ubuntu commands, which is the only option when working with an ubuntu-server. Similar problems occurred when we needed to retrieve the backup files from the aSTEP server, this were solved using a subprogram of PuTTY which is called PSCP is allow us to send or retrieve files from remote machines. The last problem that arrived was installing the gitlab version, the exact package used on aSTEP is gitlab-ce\_8.4.4-ce.0\_amd64.deb which i only usable on a amd system thus making it impossible for our group to continue this emulation. 

Alternatives to testing if the update is doable on our aSTEP server could be done in a number of other ways, here is a two recommendation that we pass along the persons that will continue on this, solution number one is to ask the It Services (ITS) to take a snapshot of the aSTEP server, a snapshot is a way of saving the virtual machines state, this enables us to restore the virtual machine state to of when the snapshot was created, by doing this we will be able to run the update on the aSTEP server and if anything goes wrong we will be able to restore the aSTEP server to is previous state. The first solution has some downfalls, the process of saving the snapshot, updating gitlab and loading a snapshot can be time consuming and while this is happening nobody will be able to saved their work on GitLab, worst case scenario would be that some group might lose the work they have done because of this. The second solution involves also using snapshot, first ask the ITS to setup another virtual and load it using the snapshot from taken from the aSTEP server, then we will be able to test if there is any complications with the update and it would not be able to effect the aSTEP server. 


\subsection{PSCP}
PSCP was used to communicate between aSTEPs server and also with our own virtual machine. the commands for PSCP is pretty straight forward, if you are using windows, you open your command prompt and go to the folder where PSCP is located or you add it to your environment, we ended up placing the PSCP.exe on the c drive. Afterwards two commands was used to either retrieve or send files like shown in 

\begin{lstlisting}
 PSCP username@host:source target
 PSCP source username@host:target
 C:\PSCP abcdef12@student.aau.dk@daisy-git.cs.aau.dk:/home/test\_folder C:\\test\_folder\\
 C:\PSCP C:\test_folder\ abcdef12@student.aau.dk@daisy-git.cs.aau.dk:/home/test\_folder
\end{lstlisting}

The first command is used to retrieve data from a remote machine and the second command is used to send. the third line shows an example of how to retrieve a file from to the aSTEP server, with the user abcdef12@student.aau.dk and the host being daisy-git.cs.aau.dk and the path on the aSTEP server being /home/test\_folder and the target path is C:\\test\_folder\\. The forth example shows the send command it is similar to the third example, with the two arguments being switched around. 


