\section{GitLab}
From the aSTEP meeting that toke place 17.02.2016 it was decided that a few people across the groups should take responsibility for maintaining the gitlab server. Two members of our group volunteered for this task. The section purpose is to describe the work that our group has done related to GitLab.

Our first task were to update the GitLab that was running on semesters virtual server, this will from now on be refereed to as aSTEP server. The magnitude of this task involves installing a virtual server on one of our groups computer to emulate the GitLab update to ensure that if it fails to update, we will not course any problems to the aSTEP server running GitLab. Early in the process we discovered some problems with this approach.

Installing GitLab were a bit more difficult then first anticipated, this were caused because GitLab run on a 64-bit architecture and the software VirtualBox only showed the option for installing 32-bit system. A solution was found that involves change the following setting on our machine; turn on Intel (R) Virtualization Technology and Intel (R) VT-d Feature in Basic Input Output System (BIOS), afterwards you need to disable Hyper-V in Windows Features and after the machine have restarted and updated the setting VirtualBox will now show the option for 64-bit.

The next step in this process is to install GitLab, we used the guide from GitLab own website\cite{gitlab_guide}, this had no major complications.

Third step is to restore the backup create from the aSTEP server, this gave some complications, but mainly duo to the inexperience working with a ubuntu-server. To extract the backup from the aSTEP server and upload it to the testing server we used sub program of PuTTY called pscp, this allow us to extract and upload file to a machine.

The last step is to update the GitLab with the new version of GitLab, this was surprisingly easy and GitLab have made a lot of measures to ensure that GitLab server will not lose data even if the update fails. This is done by automatic creating a backup before installing and detailed description if anything went wrong, and solutions to fix it.

After this was done a wiki page was created that would help future semesters in updating gitlab if necessary, the whole process of setting up af virtual test server will not be descried because it would not be necessary when GitLab have the fail safe measures.


