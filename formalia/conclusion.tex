\chapter{Conclusion}\label{chpt:conclusion}
As stated in \cref{chpt:introduction} the context of the system is a museum filled with irreplaceable objects which needs to be kept safe from intruders. 
In order to ensure the safety of the museum's possessions, a system to recognise and thereafter track an object was designed.

The requirements presented in \cref{sec:requirements} are prioritised into three groupings, where the first group, \textit{Must have}, has been fulfilled.
The two requirements in this grouping are \textit{Object Recognition} and \textit{Target Tracking}.
The \glspl{rsp} recognises and calculates an object's position in an image by using the theory described in \cref{ch:obj_rec}through the use of \gls{opencv}, thereby implementing \textit{object recognition} in the system.
The rig was built in order to provide the Lego Mindstorms motors with a base of which they can perform horizontal and vertical rotations, to move a laser around to point in specified directions in the room, in which it is located.
The \gls{nxt} controls these motors and can thereby adjust the direction of the laser, and therefore track the recognised object by aiming the laser at the object.
By doing so, the second requirement, \textit{Target Tracking}, has been fulfilled.

Since the two essential requirements are fulfilled, the project is concluded to be a success. The fulfilment has given a system for the museum's context which gives a solution for the base problem presented in \cref{sec:base_setup}.