\section{Client Server Update}
A small update has been implemented for the client server with the goal of supporting multi-server communication and paging.

\subsection*{Paging}
We previously implemented an update on the intermediate server to support paging, as described in \cref{sec:proxy_update1}. With this update in place, it is possible to select page number and page size when retrieving data. We wish to automate this process with the client server, and this update integrates code to retrieve data several times to obtain all pages. It is done by sequentially requesting data while incrementing the page number until a request returns empty because of a non-existing page. When the program reaches such a state, it assumes that it has retrieved all available data.

\subsection*{Adding servers to client}
\Cref{subsec:system_hierarchy} and \cref{sec:data_flow} describes the hierarchy and data flow between the MSE servers and the client which processes the data. 
\begin{lstlisting}[caption={Starting threads to pull data from multiple MSE servers}, label={lst:cisco_puller}, language=inc_Java]
for (CiscoPuller thread : threadList) {
    thread.start();
    System.out.println("ciscoPuller started on ip: " + thread.get_ip());
}
\end{lstlisting}
\Cref{lst:cisco_puller} showcases the code that is responsible for starting the threads that pulls data from several different MSE services. Each thread in the \textit{threadList} collection is configured to connect to an MSE service with an address, username and password. A thread will start to pull data as soon as it is started and will only stop if terminated or otherwise interrupted.

The design reflects that we wish to connect to multiple MSEs in addition to the intermediate server that we are currently using. As of now we there are no other connections between the client and other services.