\section{Intermediate server update}

\subsection*{Logging}
The intermediate server utilizes Log4j \cite{log4j} to log error messages when it is executing. These messages are stored in a log file saved locally on the server which can be accessed and read. The log only concerns errors and alike, but we believe it would be beneficial to have an additional log that records other messages as well. We would like to log requests to the server by their time, request type and request sender. That information can be used to debug potential errors and to analyse the connectivity history. 


\subsubsection*{Password}
Changing the password from being static to being dynamic was done using the programs parameters, so previouesly we had 3 parameter, IP, Cisco user and Cisco password. we add an additionally two parameter, MSE user and MSE password.

\begin{lstlisting}[caption={Failed Connection snapshot},label={lst:dynamic_password},language=inc_Java, mathescape]
public class RestfulMSE {
	$\vdots$
	private static String MSEusername;
    private static String MSEpassword;
    $\vdots$
    public static void main(String[] args) throws IOException {
    	$\vdots$
   		MSEusername = args[3];
    	MSEpassword = args[4];
    	$\vdots$
    	CiscoNetwork.GeneratePassphrase(MSEusername, MSEpassword);
    	$\vdots$
    }
}
\end{lstlisting}

In \cref{lst:dynamic_password} the new changes are shown. the new parameter are stored in line 7 and 8, These are use as parameters for GeneratePassphrase, which is used to ensure that the user has the correct username and password when they try to login.
