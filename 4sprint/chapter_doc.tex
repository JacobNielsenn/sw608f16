\chapter{Sprint 4}
This is the last sprint and will be dedicated to implementing final features discovered in \cref{cha:sprint3} and improving the overall code quality. This includes implementing some of the functionality discussed in \cref{sec:secure}, and preparing the code for handing over to the new developers. Additionally, a section will be prepared for future developers to ease the transition and make it easier to get started.


\section{Intermediate Server Update}
An update has been implemented for the intermediate server to expedite debugging and accommodate for some of the considerations made in \cref{sec:secure}. The changes are as following:
\begin{itemize}
\item Logging implemented
\item Hardcoded passwords are now program arguments
\item MAC addresses are now hashed
\item Added context to see if MSE is online
\end{itemize}

\subsection*{Logging}
During the course of the third and fourth sprint we encountered sporadic downtime on our intermediate server in the form of erroneous response of complete lack thereof. Upon further investigation, we have found that this is a consequence of MSE being down for maintenance. We have implemented two tools that will assist developers in debugging in these situations.
First, the intermediate server now utilises Log4j \cite{log4j} to log error messages during execution. These error logs are stored in log files saved locally, which can be accessed and read as cleartext. The log only concerns errors and alike, and each log entry contains a timestamp, error message, error level and exception type. We believe it would be beneficial for future development to have an additional log that records other messages as well. That information can be used to debug potential errors and to analyse the connectivity history.

Secondly, we have implemented an additional context to specifically query if the MSE is online. When requesting this URI, the system will attempt to connect to MSE and the response will depend on whether it retrieves a HTTP response code from MSE. Using this URI, it is very easy to learn that a problem stems from MSE being unresponsive, and is not a consequence of an error internally in our system.

\subsection*{Hard-coded Passwords}
In \cref{sec:secure} it was argued that login details should not be hardcoded, as anyone with access to the source code would be able to connect to the system. We have changed the password from being static to being dynamic using the programs arguments. Previously the program had 3 arguments, IP address for MSE, MSE username and MSE password. As of this update, we have added an additional two arguments, intermediate server username and intermediate server password.

\subsection*{Hashing of MAC Address}
Due to a request from the Heatmap group, to be able to track people by some unique value, we have elected to hash the MAC addresses rather than replace them with the string \emph{OBFUSCATED}. In \cref{sec:secure}, we explored the possibility of doing so, by using Javas hashCode(), which is implemented in the String library. However, hashCode() returns a string consisting of only integers, which gives a higher risk of having overlap when hashing. Instead, we chose to use the Guava-19 library, which returns a much longer value consisting of both numbers and letters. By doing so it allows for hashed MAC addresses to contain letters, reducing the possibility for overlapping hashed MAC addresses. The salt applied is a random string that is regenerated every twelve hours. 

\begin{lstlisting}[caption={Hashing a MAC address},label={lst:ourhash},language=inc_Java]
oldMacAddress = oldMacAddress + salt;
String hash = Hashing.sha256().hashString(oldMacAddress, 
StandardCharsets.UTF_8).toString();
item.setMacAddress(hash);
\end{lstlisting}

The applied function can be seen on \cref{lst:ourhash}. On line 1 the MAC address is concatenated with the randomly generated salt-value. Afterwards the new string is hashed using Guava-19's sha256, as seen on line 2. On line 4 the MAC address is set to the hashed value. The hashed value is 64 characters long and contains both letters and numbers.
\section{Client Server Update}
A small update has been implemented for the client server with the goal of supporting multi-server communication and paging. The changes made are as follows:

\begin{itemize}
\item Paging supported
\item Servers communication utilizes threading
\item Logging implemented
\end{itemize}

\subsection{Paging}
We previously implemented an update on the intermediate server to support paging, as described in \cref{sec:proxy_update1}. With this update in place, it is possible to select page number and page size when retrieving data. We wish to automate this process with the client server, and this update integrates code to retrieve data several times to obtain all pages. It is done by sequentially requesting data while incrementing the page number until a request returns empty because of a non-existing page. When the program reaches such a state, it assumes that it has retrieved all available data.

\subsection{Adding Servers to Client}
\Cref{subsec:system_hierarchy} and \cref{sec:data_flow} describes the hierarchy and data flow between the MSE servers and the client which processes the data. 
\begin{lstlisting}[caption={Starting threads to pull data from multiple MSE servers}, label={lst:cisco_puller}, language=inc_Java]
for (CiscoPuller thread : threadList) {
    thread.start();
    System.out.println("ciscoPuller started on ip: " + thread.get_ip());
}
\end{lstlisting}
\Cref{lst:cisco_puller} showcases the code that is responsible for starting the threads that pull data from several different MSE services. Each thread in the \textit{threadList} collection is configured to connect to an MSE service with an address, username and password. A thread will start to pull data as soon as it is started and will only stop if terminated or otherwise interrupted.

The design reflects that we wish to connect to multiple MSEs in addition to the intermediate server that we are currently using. As of now the client is only connected to AAU's MSE, but with this update it is now capable of communicating with an arbitrary number of servers.

\subsection{Logging}
Following the intermediate server update, we decided to implement logging on the client too. This decision has been made to facilitate debugging. It is not strictly necessary, as we have direct access to the aSTEP server, where the program is running. However, to being able to retrieve an error log whenever the program crashes is valuable for development. Consequently the client server now utilizes Log4j \cite{log4j} using the same configuration as the intermediate server.
\section{Catching Up - Introduction to Using Our Project}\label{sec:catchup}
This section will contain guides describing what elements to be aware of in the system and how to perform different tasks for maintenance and further development. We write this section to ease the change of ownership, such that the new owners can avoid problems we experienced and spent time resolving. 

\subsection*{Transfering a Program to the Virtual Server}
We recommend testing new changes as thoroughly as possible before going through this as this process is tiresome to do a lot. Before doing this you need to contact the Cisco MSE administrator of AAU and ask for passwords for the virtual proxy server and MSE. The current administrator is Per Mejdal Rasmussen from ITS. Let him know that you are continuing development of the proxy server connecting us to MSE.

Sending the Java program to the virtual intermediate server:
\begin{itemize}
\item 1) Compile the artifact such that you get a .jar executable file. Test it on your own system by running "java -jar FILEPATH" on your own system.
\item 2) Upload the .jar to a website you can download from. We used dropbox. Get the shared link on Dropbox for the file.
\item 3) Open putty and connect to "sshgw.aau.dk".
\item 4) Login with your usual moodle account.
\item 5) ssh to root@mon-mse-proxy.srv.aau.dk with the command "ssh root@mon-mse-proxy.srv.aau.dk".
\item 6) Login with password - this is something you request from the MSE administrator.
\item 7) Navigate to the home directory. This directory contains directories for current and previous versions of the program. Move current version and other files to another folder. 
\item 8) Check if any screen session is active with "screen -ls". If this is the case, kill the process with "killall screen".
\item 9) Download the .jar file with the command: wget -O MSEproxy.jar "DROPBOXURL" 
\item 9a) Make sure to put a download=1 at the end of the url if using Dropbox.
\item 10) See that you have the file in the directory, run the program and make sure it runs correctly. Command to run the program is on step 14.
\item 11) Change user to MSE with the command "su - MSE -s /bin/sh".
\item 12) Type "script /dev/null".
\item 13) Check if any screen session is active with "screen -ls". If this is the case, kill the process with "killall screen".
\item 14) Navigate to home folder containing .jar and run the program with command: "screen java -jar MSEproxy.jar https://172.18.37.70 mse-proxy PW", where PW is replaced by the password given by the MSE administrator.
\item 15) If all is well you get a message saying the server is up and running on some IP. This is the IP you connect to. The port is important as well, it has to be 8080.
\item 16) Detach from the screen by holding down CTRL+a+d.
\end{itemize}

\subsection*{IntelliJ Debug Configurations}
When further developing the project it is useful to share the debug configurations, such that the project setup only has to be done on one computer. In IntelliJ, open the Edit Configurations screen and tick the shared box. This creates a folder in the .idea project folder called runConfigurations, which can be shared. When opening the project, IntelliJ will automatically detect existing debug configurations in this folder.

\subsection*{Projects under Indoor Repository}
The Indoor (ID) Gitlab repository contains several IntelliJ projects. The projects are called: testingPrograms, RestfulMSE, indoorLibrary, indoorCore. Additionally you will be able to find a JavaDocs folder containing code documentation.

\begin{itemize}
\item Testingprograms is a simple project used to debug a RESTful service. This can equally be done with a tool in IntelliJ (under Tools -> Test RESTful Web Service), however, we do not advice this as everything has to be input each time it is to be tested. Testingprograms contains very little functionality and is only meant to be used for HTTP get calls.
\item RestfulMSE is the project for the intermediate proxy server. 
\item Indoorlibrary is a static library containing much of the functionality used by our programs. We early on realised that the two systems are alike in many ways, and would have to afford the same functionality. As a consequence we decided to make a library to use across several projects, to avoid duplicate code. 
\item IndoorCore contains the code for the client service. 
\end{itemize} 

\subsection*{First Time Opening the Program}
This is a step guide to setup the IntelliJ IDE with the project.



\section{Code Inspection}
Since this project will be further developed by other people, we find it essential to perform a final review of the code to improve readability.

By using the IntelliJ IDEA we are able to use one of the built-in tools, called \emph{Inspect code}, to analyse the code and suggest improvements. This is done for RESTsw6, indoorCore and indoorLibrary. When inspecting code, we receive suggestions in different categories: Class structure, declaration redundancy, Imports, etc. Class structure concerns suggestions related to moving global variable into a sub scope where they are used to hide information. Declaration redundancy concerns variable access level and removing redundant code such as empty methods and unused parameters. Imports relates to unused packages. We also receive notifications in regards to possible bugs as a result of constants concealed as variables.

It is important to remember that the tool only gives suggestions. Inspect code is a static review of the code, and does not perform an analysis of the code under execution. Consequently it is not able to see how a class or method is used, which sometimes results in unrealistic suggestions that are more likely to break the code.

\section{Sprint Evaluation}
To perform an evaluation of the sprint we iterate over the sprint goals and comment on whether or not they were fulfilled.

We set out to update both the intermediate server and the client server following the security considerations made in \cref{sec:secure}. These updates successfully implemented several features as well as refactored the code to streamline style and improve the overall code quality with the help of a inspection tool.

Finally, we planned on preparing a section dedicated to ease the transitioning for future developers. \Cref{sec:catchup} contains information on how to perform different tasks and solve problems that are essential for the project.