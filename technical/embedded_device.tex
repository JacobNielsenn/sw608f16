\subsection{Transfer Device}
\label{sec:rsp}
In order to collect data from a camera, prepare data for transfer and sending data to the \gls{nxt}, a device is needed to handle the communication and data manipulation.
There exists multiple devices which allow for the possibility of connecting to a camera along with manipulating and sending the data to the \gls{nxt}. The project has a focus on embedded devices, amongst these devices is the \gls{Arduino} board and the Raspberry Pi.

The \gls{Arduino} is an open source microcontroller based circuit board, for building digital devices \cite{ard_1}. The Arduino comes with the possibility of extending the capabilities of the board by additional hardware components. To enable communication with a camera over a USB connection, a USB shield is needed. The \gls{Arduino}'s extensibility also allow for network communication using an ethernet shield, such as the \gls{Arduino} Ethernet Shield 2 \cite{eth_shl_1}. An \gls{Arduino} UNO with shields for ethernet connection and USB communication costs approximately 635 DKK in retail price, not including shipping.

The Raspberry Pi is a single-board computer \cite{rsp} and is commonly used for educational purposes and embedded systems, running with a Linux based operating system. The Raspberry Pi comes with built-in USB and ethernet capabilities, which can be used for communication with both a camera and a \gls{nxt}. As the Raspberry Pi already contains the required hardware there is no need to buy additional hardware components, as opposed to the \gls{Arduino} board. However, it is necessary to buy a power adapter and as such the total cost of a Raspberry Pi is approximately 291 DKK in retail price, not including shipping \cite{rs_pi_board}\cite{rs_chg}.

The Raspberry Pi is less expensive than the \gls{Arduino} board, and is also an all-in-one solution, as such it has been chosen as the hardware component for the project. Choosing the Raspberry Pi has the extra benefit of it being significantly more computationally powerful than the standard \gls{Arduino} board, which gives more freedom when designing the system.

\begin{table}[ht]
  \centering
  \begin{tabular}{|l|r|}
  \hline
    System On a Chip & Broadcom BCM2835 \\ \hline
    CPU & 700 MHz single-core ARM1176JZF-S \\ \hline
    GPU & Broadcom VideoCore IV @ 250 MHz \\ \hline
    Memory & 512Mb SDRAM, shared with GPU \\ \hline
    Storage & None - option of connecting SD Card \\ \hline
    Network & 10/100 MBit/s Ethernet \\ \hline
    USB & 2 USB 2.0 Ports \\    \hline
  \end{tabular}
  \caption{\gls{rsp} Model B specification\cite{rsp2}}
  \label{tbl:rsp_spec}
\end{table}
