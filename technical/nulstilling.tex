\section{Fixed Starting Position}
Now that the rig structure has been determined, the next logical step is to consider how the rig can maintain information about its own position in the global coordinate system. 

To ensure that it is possible to track the pointer's exact position at a given time, it will be necessary to maintain an internal representation of the current position. However, this does not by itself allow tracking as these values will be different from one run to the next with the same physical positioning, as primary memory is wiped when the hardware is turned off. As a consequence it is necessary to calibrate the values at start-up, such that the physical positioning corresponds to the internal representation.

There are several ways to perform this calibration. First, the values can be written and stored in the non-volatile secondary memory of the \gls{nxt}. Using this method the \gls{nxt} can save the values before shutting down and read them on start-up. However, it has the disadvantage that it requires system alterations, such that a save function is executed before the system is shut down. Alternatively the values can be stored exclusively in secondary memory, where they are refreshed and updated in short intervals when the motors move. This is very costly and is expected to have severe consequences for the real-time requirements for the system. This method does not directly perform the calibration, but instead avoids having to do it. Choosing to follow this method leads to several potential issues. For example, it is difficult to specify how initial calibration is performed. At the same time the set-up must not be moved when turned off, as this will cause the internal presentation to deviate from the physical presentation. The latter issue renders this method very difficult to properly execute, and alternatives are therefore considered.

Another option is to install and use designated hardware to measure the angles at which the pointer is turned on start-up. This could have an impact on the rig, as more weight is added. The cost of this method extends to the time it takes to properly integrate such a sensor.

A third option is to manually reset the pointer by physically adjusting it to a predefined fix-point. In order for this method to be as precise as the former options, there has to be structures on the rig supporting it. In practice this means building a simple lock that the pointer can be manually adjusted to, and lock into, such that the angle of the pointer on start-up always is the same. The disadvantages of this is the manual procedure required at every restart, together with the imprecisions connected with the manual procedure.

The most advantageous method for the project is to choose the third option, and spend additional time testing the lock for imprecisions. The work required to perform a manual reset is assessed to be a lot cheaper than storing information in non-volatile memory and the potential challenges this could lead to.