\section{Inter-device Communication}
In the previous sections the \gls{nxt} and \gls{rsp} were chosen as the hardware for communicating between the laser unit and camera units. This system will consist of a single \gls{nxt} and several \glspl{rsp}; one for each \gls{cvbw3}. The \gls{nxt} needs to get information from every camera that is able to see the target in order to point its laser towards it. This section evaluates the options for communication between the camera units and the laser unit.

In order to minimize delay and uncertainties it has been chosen to use a wired connection, which excludes a Bluetooth connection. The available alternatives for wired communication for the \gls{nxt} are the RS485, I\textsuperscript{2}C, and USB protocols\cite{lego_hardware}.

The RS485 protocol is only supported by the \gls{rsp} if an extra shield is used\cite{rs485_shield}. This makes it a less preferable solution as additional shields are needed for every \gls{rsp}.

The I\textsuperscript{2}C protocol is a master-slave protocol that can be used both by the \gls{rsp} and by the \gls{nxt}. However, neither of the two devices can be slave. The \gls{nxt} does outright not have the option of being slave\cite{lego_hardware} and if the \gls{rsp} is to be slave, two GPIO pins are needed; one for the data transfer and one for clock reference. The pin for the clock is not connected on Model B\cite{pi_gpio_1}\cite{pi_gpio_2} so this protocol is not an option since the \gls{rsp} Model B is used for this project.

The last option is a USB connection as both the \glspl{rsp} and the \gls{nxt} have USB-ports. Lego has released a Windows driver for the \gls{nxt}\cite{lego_driver}. However, with only 512mb of flash memory it has been chosen not to install a Windows OS on the \glspl{rsp} and therefore it is not a solution to directly connectet a \gls{rsp} with a \gls{nxt} by USB.  

None of the three options can be used to directly connect the \gls{nxt} to the \glspl{rsp}. The I\textsuperscript{2}C protocol can not be used with the chosen hardware while the two other options needs additional hardware. Since no protocol is supported by both the \glspl{rsp} and the \gls{nxt}, an intermediate component must be integrated. One possibility is to have a laptop to act as a modem between the \glspl{rsp} and the \gls{nxt}. Acquiring a laptop, Ethernet cables and a switch is considered to be easier than ordering RS485 shields for the \glspl{rsp} which made it an easy solution for the inter-device communication. If the computer is connected to a switch with Ethernet, all \glspl{rsp} can connect to the same switch through Ethernet cables, and as such the devices can communicate. The laptop will be running Windows since it can use the released USB driver and thereby connect to the \gls{nxt}. 