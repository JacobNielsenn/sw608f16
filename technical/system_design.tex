\section{System Design}
\label{sec:system_design}
The \glspl{rsp} model B and \gls{nxt} has been chosen as the hardware to control the laser unit and camera units. For these units to be able to communicate, an intermediate device is integrated in the system. This device is a laptop running a Windows OS which uses USB to communicate with the \gls{nxt} and Ethernet to communicate with the \glspl{rsp}. This section explores the possibilities for the dataflow and at which points calculations can be performed in order to provide a solution for the problem presented in \cref{sec:base_setup}.

It has been chosen to keep all calculations outside of the computers scope. This decision has been made because the computer is intended as a means of communication. By simply letting it be a modem, one could in theory use any device capable of running a Windows OS, and using the USB driver, for the task of communicating data. This intermediate device can be chosen regardless of its computational capabilities as no computational power is needed.

There are two key calculations that have to be performed in the system. Firstly, every image captured with the \glspl{rsp} have to be analysed, and secondly, the results have to be gathered and used for calculating where the laser has to point to. It is obvious that these two calculations have to be done in sequence; it does not make sense to attempt calculating where the laser has to point without having something to point at. It was discovered in \cref{sec:rsp} that the \gls{rsp} is computationally powerful, and comparing the specifications of the \gls{rsp} with the \gls{nxt}, as seen in \cref{tbl:rsp_spec} and \cref{tbl:nxt_spec}, reveals that the \gls{rsp} is significantly more powerful than the counterpart. It is therefore more fit for the task of analysing images than the \gls{nxt}. Performing this calculation on the \glspl{rsp} also means less data have to be transferred, as the size of the image analysis result is smaller than the size of the image. The \gls{nxt} can then collect the data and calculate where it needs to point to.

Alternatively the \gls{nxt} can perform both calculations. However, doing this results in more data being communicated between the devices, as every \gls{rsp} in the system has to transfer an image to the \gls{nxt}, and ultimately there is a risk that the microcontroller on the \gls{nxt} is too slow at performing the tasks and performance will be degraded. The amount of data transferred increases significantly as more camera units are integrated in the system. This increase is much smaller if the first task is performed immediately on the \glspl{rsp}.

On the basis of the limitations of the \gls{nxt}, and the amount of data that needs to be communicated, it has been decided to use the \gls{rsp} for performing the image analysis. The results are then transferred to the \gls{nxt}, which uses these results to calculate where it has to point at.
