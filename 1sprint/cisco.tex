\section{System for Monitoring Electronic Mobile Devices}\label{sec:monitoring}
This section will describe a solution for the problems associated with indoor navigation and monitoring of multiple mobile devices.    

\subsection{Indoor Monitoring Obstacles}
When monitoring electronic devises in indoor spaces, a set of obstacles are presented. Some of the popular techniques that can be used in outdoor environments prove themselves obsolete or severely hindered when used in the indoor environment. This can be seen when working with Global Positioning System(GPS) technology as walls, roofs and floors will disrupt the signal received from satellites. As it is often experienced in a cement basement when trying to connect to a GPS.

There are ways to overcome the GPS issue and Google presents one possible solution. They have made indoor navigation and tracking possible in several indoor spaces trough their Indoor Maps project which lets users map a building by uploading a floor plan of said building. After the floor plan have been accepted, the system will request information about the location of the Wi-FI access points in the building. By doing so the system is able to triangulate with the access points and combine that information with the GPS location, thereby utilising both technologies to get a more precise position of the device than by using GPS only.

Another solution is be to utilise Radio Frequency Identification(RFID) to create checkpoints throughout a building. These checkpoint are placed in doors or other room entrances and thereby utilise a RFID scanner that reports any observed devices which is equipped with a RFID tag\cite{indoor_bin}. %The environment of which the system will be used in have to be modelled in order to make use of the checkpoint data. 
This approach has two major downsides. The first lies in the need of hardware as the environment which the system is to operate in are not equipped with the scanners, which will have to be installed and modelled. The second downside is the limitations of the system. It is only possible to get new data about a device's position when it has been observed at a scanner, which means there will be much downtime if the device is moving down a hallway or is stationary in a room. It is therefore not possible to get a precise position of a device after it has moved away from a scanner due to the direction in which the device moved in is unknown. A solution to this problem could be to use more scanners or combine the system with more technologies as previously suggested at the Google system.

There have also been done resurge on Cisco. Cisco has developed a location aware wireless network system, the Cisco Unified Wireless Network(UWN), which utilises Wi-Fi triangulation\cite{CiscoTri} and other wireless communication protocols in order to develop other indoor location based services\cite{uwn}. 
The building is modelled by the use of an top-down image of the floor plan. The image will be converted to a coordinate system that is placed on top of the model which is then supplied with the positions of the access points that are set up in the building. From there it is possible to get the relative position of any nearby devices in relation to the origin of a coordinate system.It is these positions that will be used in order to track and analyse people.

\subsection{Cisco}\label{subsec:cisco}
At the university we have access to a sever that runs the Cisco UWN software and is set up to monitor every building of the campus at Aalborg University. We have chosen to use the data that is available from the here form as it is comprehensible to set the system up in a new environment.

% billede af hvordan det ser ud?
%http://www.cisco.com/c/dam/en/us/td/i/200001-300000/270001-280000/276001-277000/276195.tif/jcr:content/renditions/276195.jpg

%How do we solve the previous stated problems