\chapter{Future Work}
\label{Cha:Future_Work}
In this chapter we describe what we believe will be advantageously for future students to look into in order to improve and further the system 
%This chapter list the aspect of Cisco system and Indoor location based service work, for future semesters.

\begin{itemize}
	\item Research better ways of obfuscating the MAC addresses so it is not stored in plain text.\ofx{have we not done this sufficient now?}
	\item Secure the connection between Cisco system and the DB server, so data is not possible to sniff.
\end{itemize}


\section{Data Flow from Cisco to the Database}\label{sec:data_flow}
\begin{figure}[ht]
	\begin{center}
		\includegraphics[scale=0.7]{graphics/ciscoNew.pdf}
		\caption{Cisco systems}
		\label{fig:cisco_systems}
	\end{center} 
\end{figure}
\Cref{fig:cisco_systems} shows how the information flow is intended. The client connecting to the Cisco services can function as a server that requests data from all Cisco services that we have access to. Alternatively, a server can be implemented for each Cisco service. However, this solution has several downsides. First, it means corporations supplying the aSTEP project with information will have to have hardware running the server, which requires maintenance. Second, this solution is not scalable. The server running the aSTEP database will potentially be overloaded, as it has to handle each individual Cisco system sending information. As more Cisco systems are added, there will be less time to process received data. An alternative approach is to have the aSTEP database server request data from the mobile apps, however, with an increase in connected Cisco systems, the interval at which we receive new information from a given system also increases. The optimal solution to this issue is to create a hierarchy of servers, such that the database server only receives data from a constant amount of intermediate servers, each of which also receive information from an amount of sources. During the start up of the aSTEP project we have access to a single Cisco MSE system, and as such we will not focus on building this hierarchy. However, as the project grows and additional Cisco systems are integrated, it will eventually become a necessity.

\section{Further development on intermediate server}\label{sec:future_intermediate}
As we are working on developing these tools to gather data, Cisco are further developing MSE. It is likely that new functionality has been added once development continues on this project, and it is therefore suggested that MSE is updated. The documentation for the newest version can always be found online, including the RESTful api documentation. From what we experienced, the documentation for the RESTful api is very poor, and requires additional effort to test functionality and acquire a good understanding of the api. Newer versions of MSE have some features that initially appear useful, but require a closer look to determine whether the project can make use it. We have chosen not to put effort into examining these features, as the documentation is very poor and easily misunderstood, and we have had very limited access to a system where we could test functionality.

One slightly obscure feature that MSE offers is to obfuscate MAC addresses. We do not know whether this is a binary setting, or if it is possible to configure this in more advanced ways. Similarly, MSE also offer the ability to subscribe to movement, such that subscribers are notified when movement occurs. However, this requires additional configuration on the MSE side, as well as a server that MSE can notify. Lastly, we suspect it might be possible to simply request positioning for the devices that have changed positions recently. We suggest these features are further explored.

\section{The Future of Our IPS}\label{sec:futureSystem}
As described in \cref{sec:ourSys} we have been working on building our own IPS. However, it was not possible for us to finish it as we encountered a significant difficulty. The obstacle we encountered was not being able to acquire the RSSI, and thus making it impossible for us to make further meaningful progress on our system within the remaining time limit. We do however believe it would be a fitting and interesting task for future developers.

The first step, if someone is to furtherer develop the system, should be to acquire the RSSI for devices connected to the network. The next step should be to expand the range of attainable devices by allowing multiple access points. From there is should be possible to calculate the distance between each access point based on the signal strength. This calculation can be performed by utilising one of the methods described in \cref{sec:tracking_approach}. By implementing these features it would be possible to calculate the positions of devices on the network, which we believe would be a crucial component of creating the system.

If the described features were implemented, a set of secondary functionality could be considered. The system could be expanded to receive the signal strength for all devices within the networks perimeter, connected or not. This would allow the system to track everyone within its coverage and thereby give it a more accurate view of the area.
Another beneficial feature the system could support, is be to calculate an estimate of how likely it is that the found locations are correct. This estimate could be similar to that the MSE provides with its \textit{confidenceFactor} described in \cref{sec:class_design}. This estimate will be useful for comparing the system's precision with other systems as descried in \cref{sec:ourSys}. 

\section{Coordinate Formates}
As described in \cref{sec:geo_coordinates}, there are three formats for geographical coordinates. We have chosen to provide the decimal degree format because it is the format used by the outdoor location groups and the format requested by the application group utilising indoor positioning.
However, future applications as well as future development on existing applications may need other formats. We have in \cref{sec:geo_coordinates} provided the necessary calculations to change the format. These calculations can be used in the future to implement the api.
\section{Security}\label{sec:fucture_secure}
More features could be implemented to further security in our system, as discussed in \cref{sec:secure}. We believe this is something for future students to focus on, and to implement as we handle sensitive information which should be handled in the best possible we.

\subsection{HTTPS on Intermediate Server}
We believe the most significant improvement future students can make to further security in this system would be to implement HTTPS on the intermediate server.
\section{Integration Test}\label{int_testing}
As of now the tests are made as unit tests, as described at \cref{subsec:unit_testing}, which is insufficient because of how the components of the system interacts.

An integration test is multiple unit tests that are aggregated to form a component \cite{msdn_it}. This type of tests would be beneficial for the system as all of the network communication methods are heavily intertwined within each other. This is apparent when analysing the data flow throughout the components that utilises these methods. It is possible to continue using the JUnit test suite for these tests, which means that one should not be weariful % weariful--> worryful/fearful
of conflicting test environments.

One could argue that instead of relying on integration tests, it would be better to refactor the methods and components such that they are more atomic. By doing so, it would be easier to create likewise atomic unit tests that accompany the new refactored methods. This would now remove the need of integration tests as they are th next logical step in the software's life cycle as they are useful for testing major parts of the system. 


%Har vi husket at nævne de manglende unit tests i konklusionen?w