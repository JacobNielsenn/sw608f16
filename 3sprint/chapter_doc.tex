\chapter{Sprint 3}\label{cha:sprint3}
This sprint aims to explore the possibilities of building our own tracking system as well as updating the intermediate server to finally fulfil the goals from \cref{cha:sprint2} and resolve issues. Furthermore, we will examine our system regarding basic methods to increase security.

\section{Intermediate Server Update}\label{sec:proxy_update1}
We plan on implementing an update to the intermediate server that is used to communicate with AAU's MSE. This update is a consequence of MSE being updated to support geographic coordinates, as mentioned in \cref{sec:geo_coordinates}. The update is planned to remove a bug causing loss of data during conversion between different data formats. This section covers the motivation for the following changes in the update:

\begin{itemize}
\item Resolved loss of data during conversion
\item Obfuscates new data acquired as a result of aforementioned change
\item The service no longer stops if it loses its connection with MSE
\item Now checks if MSE is online, and whether the right account information is received, at start-up
\item Supports paging
\end{itemize}

\subsection*{Loss of Data}
Geographical coordinates are now supported by MSE, thanks to the MSE administrator at AAU. To perform obfuscation of personal sensitive data, we can convert the returned string to classes containing all information in its member variables. With this method we can easily perform obfuscation of the data we are disallowed to keep by editing the fields corresponding to the personal sensitive data.

This method utilises a library developed by Google called gson \cite{gson}, which affords the ability to convert strings of the JSON format to a Java class and vice versa. We can acquire the classes by using an online tool called jsonschema2pojo \cite{jsonschematwopojo}, that analyses JSON strings and automatically generates classes with corresponding fields.

Initially, the generated classes did not contain fields for the global coordinates, as they were based on the response string from a version of MSE where this feature was disabled. This would not be a problem for a single conversion, as the generated classes have a field called \textit{additionalProperties}, which contains additional information that could store the coordinates. However, in order to preserve the transparency of the intermediate server, we wish to perform a conversion back to JSON in order to maintain the external illusion of this intermediate server being an MSE service.

Performing several conversions results in a loss of data, as any information in the additionalProperties field is lost. To avoid this data loss, we can re-generate the classes using a JSON string with the appropriate information. Alternatively, we could perform a string search for specific keywords in the JSON string, which we would then obfuscate. However, search and replace in strings is undesirable, in particular because the JSON string often is more than a million characters long. The class of strings in Java is immutable, meaning any changes to a string variable results in new memory being allocated and the modified content of the old memory location copied over. Performing replacements in a large string is very expensive, and it is as such desirable to perform the aforementioned class conversions. 

As a result of the update we now receive specific information that were previously unattainable through the MSE developer test system. This information includes personal sensitive data, which is now being obfuscated when necessary needed. 

\subsection*{Connectivity with MSE}
After the initial deployment of the service we experienced downtime as a result of MSE being under maintenance. This was an unexpected event, and as such the program reacted in an unforeseen fashion. Part of the update integrates elements from the client, described in \ref{sec:fetch_data}, in an effort to increase robustness and as such uptime of the service. As a result, the service now sends an initial request to MSE, to confirm the MSE IP address and account info. Furthermore, the service is able to handle MSE being unresponsive.

\subsection*{Paging}
When requesting all clients from MSE the response is partitioned into pages. Each page contains a certain amount of entries, as defined in the request. If this is not defined in the request, a default of 5000 entries per page is used. 

Initially we were not aware of this feature, as the MSE test service never made use of paging. As a result, we would only receive the first page of information, missing out on thousands of entries. This update aims to implement a correction, which will enable the use of paging in a request. After this update it is possible to explicitly state which page is requested, whenever a request is sent.
\section{Building our own tracking system using routers and hotspots}
This section will cover the advancements made in building and gathering information about routers and hotspots to make our own Indoor Location-Based System (ILBS). A list of objectives can be seen below. The primary goal is enabling us to calculate our position in geo-coordinates, in order to compare it with the coordinates provided by Cisco MSE.

\begin{itemize}
	\item Equipment Requirements
	\item Test Capabilities of the Router 
	\item Test Capabilities of the Access Points
\end{itemize}

\subsection*{Equipment Requirements}
The initial requirement for the equipment is that it comes with a wireless antenna supporting 2.4 GHz. It would be advantageous if we set up two different ILBS for the sake of comparison. We decided to buy two brands, allowing us to analyse two ILBS and thus conclude which of the systems performed best, based on the criteria described in section \ref{sec:monitoring}.

Based on the above requirement we found two brands, Ubiquiti and D-link, the latter of which is considered the cheaper brand. We ordered one router, two expensive access points and five cheap access points from each brand.

\subsection*{The physical setup}\ofx{we may want to move this?}
Cisco\cite{access_point_placement} have guidelines for how to place the access points in order to get the best coverage for your system. They recommend that there is an access point in each corner of the building and some along the perimeters. In addition, if the building is large enough, access points should be placed within the perimeter to form a sub-perimeter which itself may contain a sub-perimeter and so on. This is illustrated in \cref{access_placement}, in which the red circles are access points. The perimeter established by the outer-most access points should encapsulate the entire floor, as illustrated by the blue line and the sub-perimeter by the red line. Everything on this perimeter is called the \textit{hull}\cite{access_point_placement}.

Each access point should be place 50-70 feet away from each other in order to get a high precision without interference and unnecessary overlap\cite{access_point_range}.

\begin{figure}[H]
	\centering
	\includegraphics[scale=0.5]{graphics/access_placement.png}
	\label{fig:access_placement}
	\caption{Illustration of how access points should be placed\cite{access_point_placement}.}
\end{figure}

An overview of our system setup can be seen in \cref{fig:OwnSetup}. This setup follows the previous design guidelines. One router is set up in the middle and four access points are set up in our group rooms, such that it covers three group rooms.

\begin{figure}[H]
	\centering
	\includegraphics[scale=0.5]{graphics/Router-AccessPoint_Setup.pdf}
	\label{fig:OwnSetup}
	\caption{Illustration of how a positioning system can be set up, using routers and access point that can track devices.}
\end{figure}

\subsection*{Test Capabilities of the Equipment}
We tested the Ubiquiti equipment, with the original firmware. At first glance we are able to get transmission(TX) and receive(RX) signals of connected devices. Unfortunately we do not receive the signals from devices not connected to the network. It was attempted to use the routers terminal which did not yield any results due to restricted access.

This caused us to research custom firmware for the router. By installing new firmware on the router we will be able to get root access to the router, and thereby manipulate it at a lower level than the routers original firmware allows. This is necessary if we are to capture every RX signal the router gets.

We did find custom firmware for Ubiquiti, however not for D-link. The searched sites were: openwrt.org\sfx{cite?}, polarcloud.com\ofx{cite?}, dd-wrt.com\sfx{cite?}, gargolye-router.com\ofx{cite?}, librecmc.org\ofx{cite?} and wrtrouters.com\ofx{cite?}. These sites cover thousands of routers and the fact that we were not able to find a custom firmware for D-link might be because the particular model we ordered\ofx{something is missing} because other routers from the same brand is supported. No conclusion was found as to why it is not covered on any of the websites.

We ended up utilizing the openwrt because it supports our Ubiquiti router and allows us to gain root access and execute programs. With the new custom firmware installed there is roughly 8Mb of free memory left on the router. This put some limitations on which language we can use. Our first attempt is to use a version of python called Mini-Python that does not exceed the memory limit.

A program is constructed allowing us to listen on different types of networks such as LAN and WAN. This is done using the socket library which is a standard library in Python. However, it is not part of the minimalistic version, meaning we had to import it ourselves. We were not able to do this either due to space consumption.

/jfx{Nuværende konklusion, vi kan fange pakker, hvis devicen er forbundet til vores netwærk eller de selv har tænkt for internet deling. Det er dog ikke muligt at få RSSI (signal styrke), der er ikke fundet nogen defintive årsag til dette, men formoder at det kan skyldes driverne på routeren (openwrt).}
\section{Security considerations}
In order to increase security in our system, different considerations was made. These considerations are made in order to make it more difficult for outsiders to access the system as well hide sensitive user information except for the mac-address.
\ofx{tie this to sprint1 security}
\subsection*{HTTPS and Certificates} 
%  HTTPS and SSL certificate. May want to talk more about where we could use it in our system
To increase security against unauthorised personal a HTTPS connection could be established. HTTPS is short for \textit{Hyper Text Transfer Protocol Secure} and is the secure form of HTTP. With HTTP everything send as plain text and can be read by anyone who access the connection, but what makes HTTPS secure is that this pure text is encrypted. This means that if someone gains access the connection they would not be able to read the information. A secure connection is therefore very important on sites that handles sensitive information such as CPR-numbers or credit card information and this can be achieved by using HTTPS\cite{HTTPS}.

\begin{figure}[ht]
	\begin{center}
		\includegraphics[scale=0.9]{graphics/HTTPS.png}
		\caption{HTTPS \cite{https_pic}.}
		\label{fig:HTTPS}
	\end{center} 
\end{figure}

\Cref{fig:HTTPS} shows how HTTPS works after the key exchange have taken place. First the sender sends plain text, which is encrypted using the recipients public key. The encrypted message is then send and decrypted using the recipients private key before plain text is received.

The handshake between the browser and a site using HTTPS starts with an exchange of public SSL-keys (Secure Socket Layer). The SSL-key is unique and enable encrypt of the data you send, the encrypted data can then only be decrypted with your own private key\cite{HTTPS}. 

\subsection*{Hashing}
Hash function are most commonly used to store a users password or other sensitive data. In functions as a one way function, when a message is hashed it can not be reversed again. The formal definition is shown below:
\begin{quote}
\textit{'A hash function is a computationally efficient function mapping binary strings of arbitrary length to binary strings of some fixed length, called hash-values.'\cite{Hash_def}}
\end{quote}

It have been requested by the heatmap application group to have a unique identification of all people. If we are not allowed to store the MAC-Address, a solution to this would be to hash the MAC-Address with a random, daily changing salt. In this way it will be possible for the heatmap to track people for as long af the salt-value is unchanged.

Java have a hash-method implemented in the object class called \textit{hashCode()}, this may be used for hashing information. When using a hash function properly, you add a random generated string that extends the original message. The generated string is called a salt and make the hash function more secure. An example of how to apply the Java hashCode can be seen in \cref{lst:hash}.
\begin{lstlisting}[caption={Hash a MAC-Address},label={lst:hash},language=inc_Java]
String salt = UUID.randomUUID().toString();
hash = MAC\_Address + salt;
hash.hashCode();
\end{lstlisting}
In \cref{lst:hash} we see that a random string is generated in line one, then concatenated with the information we wish to hash, in this case the MAC-Address. Finally we call the Java hashCode() on the string, and this gives us a string that can not be traced back to the original.

\subsection*{Hard-coded passwords}
In our connection to Cisco as well as to the intermediate server we are currently using hard-coded passwords, which means that everyone with access to the code have the log-in information. These should be hashed and stored such that the log-in can not be seen in the code and is never seen as plain text. 

\subsection*{Visibility}\ofx{may be moved to refactoring}
As of now the most of our system is public, by decreasing the level of visibility to have more private and protected classes and methods, the system would be more secure.
 
\subsection*{Obfuscate personal information}
In order to ensure our users personal information, it have been considered to obfuscate their E-mail as well as their Ip-addresses in the same manner as it is done to the MAC-address. By doing so there will be no way for outsiders to identify people unwilling to have their information stored.


\section{Sprint Evaluation}
To perform an evaluation of the sprint we iterate over the sprint goals and comment on whether or not they were fulfilled.

During this sprint we examined the possibility of building a small tracking system utilising access points. The goal was to have a system that is comparable with MSE in terms of the precision of the geographical coordinates. We were not able to make a functional IPS as we were unable to retrieve the RSSI from the routers and access points. \Cref{sec:futureSystem} will function as a guidance and proposal section for how to continue the development of the system.

We deployed an update for the intermediate service communicating with MSE, fixing several issues and creating support for geographical coordinates opposed to the relative once we got before. The update was successfully installed and deployed. This concludes our shortcomings from \cref{cha:sprint2}.

The security considerations lead us to be wary of different security pitfalls and sensitive areas of the system. We have made some decisions regarding what information to obfuscate and what information to keep. It was decided to obfuscate the MAC address and e-mail, by hashing the MAC address and deleting the e-mail address.