\section{Restful Proxy Update}\label{sec:proxy_update1}
We plan on implementing an update to the restful proxy server that is used to communicate with Aalborg University's MSE. The update is planned to remove a bug causing loss of data during conversion between different data formats. This section covers the motivation for the following changes in the update:

\begin{itemize}
\item Fixed loss of data during conversion\afx{Det ville vaere bedre hvis det var mere formelt. Undgaa at bruge Fixed}
\item Obfuscates new data acquired as a result of aforementioned change
\item The service no longer stops if it loses its connection with MSE
\item Now checks if MSE is online, and the right account information is received, at start-up
\item Supports paging
\end{itemize}

\subsection{Loss of Data}
Global coordinates are now supported by MSE, thanks to the Cisco administrator at the Aalborg University. To perform obfuscation of personal sensitive data, we can convert the returned string to classes containing all information in its member variables. With this method we can easily perform obfuscation by editing the fields corresponding to the personal sensitive data.\afx{Naevn at det ikke er alle clienter som vi obfuskere, men kun dem hvor vi ikke har tilladelse til gemme deres informationer}

This method utilizes a library developed by Google called gson\cite{gson}, which affords the ability to convert strings of the JSON format to a Java class and vice versa. We can acquire the classes by using an online tool called jsonschema2pojo\cite{jsonschematwopojo}, that analyses JSON strings and automatically generates classes with corresponding fields.

Initially, the generated classes did not contain fields for the global coordinates, as they were based on the response string from a version of MSE where this feature was disabled. This would not be a problem for a single conversion, as the generated classes have a field called \textit{additionalProperties}, which contains additional information that could store the coordinates. However, in order to preserve the transparency of the intermediate server, we wish to perform a conversion back to JSON in order to maintain the external illusion of this intermediate server being an MSE service.

Performing several conversions results in a loss of data, as any information in the additionalProperties field is lost. To avoid this data loss, we can re-generate the classes using a JSON string with the appropriate information. Alternatively, we could perform a string search for specific keywords in the JSON string, which we would then obfuscate. However, search and replace in strings is undesirable, in particular because the JSON string often is more than a million characters long. The class of strings in Java is immutable, meaning any changes to a string variable results in new memory being allocated and the modified content of the old memory location copied over. Performing replacements in a large string is very expensive, and it is as such desirable to perform the aforementioned class conversions. 

As a result of the update we now receive specific information that were previously unattainable through the MSE developer test system. This information includes personal sensitive data, which is now being obfuscated when necessary needed. 

\subsection{Connectivity With MSE}
After the initial deployment of the service we experienced downtime as a result of MSE being under maintenance. This was an unexpected event, and as such the program reacted in an unforeseen fashion. Part of the update integrates elements from the client, described in \ref{sec:fetch_data}, in an effort to increase robustness and as such uptime of the service. As a result, the service now sends an initial request to MSE, to confirm the MSE IP-address and account info. Furthermore, the service is able to handle MSE being unresponsive.

\subsection{Paging}
When requesting all clients from MSE the response is partitioned into pages. Each page contains a certain amount of entries, as defined in the request. If this is not defined in the request, a default of 5000 entries per page is used. 

Initially we were not aware of this feature, as the MSE test service never made use of paging. As a result, we would only receive the first page of information, missing out on thousands of entries. This update aims to implement a correction, which will enable the use of paging in a request. After this update it is possible to explicitly state which page is requested, whenever a request is sent.